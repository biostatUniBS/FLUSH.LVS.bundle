%\VignetteIndexEntry{FLUSH: filtering genes}
%\VignettePackage{FLUSH.LVS.bundle}
\documentclass[12pt,a4paper]{article}
\usepackage{amsmath,pstricks}
\usepackage{hyperref}
\usepackage{url}
\usepackage[authoryear,round]{natbib}

\textwidth=6.2in
\textheight=8.5in
%\parskip=.3cm
\oddsidemargin=.1in
\evensidemargin=.1in
\headheight=-.3in


\newcommand\Rpackage[1]{{\textsf{#1}\index{#1 (package)}}}
\newcommand\dataset[1]{{\textit{#1}\index{#1 (data set)}}}
\newcommand\Rclass[1]{{\textit{#1}\index{#1 (class)}}}
\newcommand\Rfunction[1]{{{\small\texttt{#1}}\index{#1 (function)}}}
\newcommand\Rfunarg[1]{{\small\texttt{#1}}}
\newcommand\Robject[1]{{\small\texttt{#1}}}


\newcommand{\scscst}{\scriptscriptstyle}
\newcommand{\scst}{\scriptstyle}
\author{Stefano Calza \\ {\tt calza@med.unibs.it} \\ \url{http://www.biostatistics.it}}
\usepackage{Sweave}
\begin{document}
\title{Filtering genes using FLUSH}
\maketitle
\tableofcontents





\section{Introduction}

This document describes the flow-chart procedure for filtering genes
in an experiment based on Affymetrix expression microarray data.

The FLUSH method for filtering genes is based on robust linear models
fitted at the probe level. 


After starting R, you should load the \Rpackage{FLUSH.LVS.bundle}
package.

\section{Fitting the robust linear models}

The first thing to do is get an object of class \Rclass{AffyBatch}.
This can be done reading in CEL files using the function
\Rfunction{ReadAffy} in the \Rpackage{affy} package. In this vignette
we will use the data from Choe et al., available as an R data object
here \url{http://www.biostatistics.it/library/FLUSH.RData}.

To read the data in R we can do as follows, where we assume that the
``choe.RData'' file is in the working directory.

\begin{Schunk}
\begin{Sinput}
> load("FLUSH.RData")